\chapter{Large-scale datasets for Switzerland}
\label{data}

% SOURCE: PV paper (original draft)
% Using big data mining techniques for the estimation of large-scale RPV potential requires the availability of accurate and high-resolution environmental and building datasets. Spatial and temporal resolutions of these data are rapidly increasing, but the highest-quality information is frequently not available for a large spatial coverage. Switzerland has been selected as case study area for this approach, as large amounts of data at different resolutions and spatial coverage exist for solar radiation, the building stock and digital surface models.

The spatio-temporal modelling of solar photovoltaic and shallow geothermal energy potentials at national scale for Switzerland requires the availability of high-resolution data on the existing building stock, meteorological data and data on the subsurface geology. 
As most of these data are used in several instances and play an important role in the choice of suitable modelling approaches, the relevant large-scale datasets used throughout this thesis and relevant pre-processing steps are detailed in the following.

\section{Building and landscape data}
A large amount and variety of building and landscape data is available for Switzerland. These include statistics related to the existing building stock and their use for the residential and service sector, the geometries of building footprints, rooftops and other built-up areas, as well as high-resolution digital elevation models (see Table~\ref{tab:bld_landscape}). Most of this data can be accessed freely on the mapping platform of the Swiss Confederation\footnote{\url{https://map.geo.admin.ch/}}.

[ADD TABLE HERE]

\subsection{Building and enterprise statistics}
\label{data_rbd_statent}
\textbf{Registry of Buildings and Dwellings (RBD).} The RBD, published and continously updated by the Swiss Federal Statistical Office (FSO) \cite{bundesamt_fur_statistik_bfs_eidgenossisches_2015}, contains 1.6 million registered buildings in Switzerland.
It lists a large amount of building-related attributes, including coordinates, floor area, construction period, number of floors, number of dwellings, and information on the building use (e.g. residential/service sector/industrial). 
The RBD further contains a registry of 4.2 million [CHECK] dwellings, amongst other listing the dwelling's area, number of rooms and dwelling use.
\nomenclature[A]{RBD}{Registry of Buildings and Dwellings}

As the attributes in the RBD are missing for some buildings and dwellings, we estimate the missing characteristics in order to estimate RE potentials for the entire building stock. [COMPLETE WITH DETAILS]

\textbf{Structural Business Statistic (STATENT).} In addition to the RBD, the STATENT provides information on the number of enterprises and employees in each of the service and industrial sectors, at a resolution of $100 \times 100 m^2$ pixels for Switzerland \cite{bfs_statistik_2018}. The sectors are divided according to the Swiss General Classification of Economic Activities (NOGA) into 88 NOGA codes, which we group to 19 sub-sectors following the classification proposed in \cite{bfe_energieverbrauch_2019} [APPENDIX?].
\nomenclature[A]{STATENT}{Structural Business Statistic}
\nomenclature[A]{NOGA}{General Classification of Economic Activities}

\subsection{Building and landscape geometries}

\textbf{Building footprints.} are available at different 
% SOURCE: PV paper (original draft)
% We compute the RPV potential per roof surface, based on a national-scale dataset of building rooftops.

\textbf{Parcel geometries}
The parcel boundaries are vector representations of the official mensuration data for around 100,000 property units, obtained from the cantonal geoinformation services of Vaud (ASIT-VD) \cite{asit_vd_cadastre_2019} and Geneva (SITG) \cite{sitg_parcelles_2020}.

\textbf{Roof surfaces.}
The dataset contains 9.6M vector polygons which are derived from a national 3D building model (LOD~2) by the Swiss Federal Office of Topography (SwissTopo) \cite{klauser_solarpotentialanalyse_2016}.
The roof polygons represent the roofs of 3.7M buildings in Switzerland and contain information  on the tilt, aspect and tilted area of each roof.

\textbf{Roof superstructures.}
In the Canton of Geneva, where detailed city GML data (LOD~4) exists, an additional dataset of roof superstructures is available through Geneva's geographic information system (SITG)~\cite{sitg_superstructures_2019}. %NEED TO ADD SOURCE
These are vector polygons which represent objects on rooftops such as dormers and chimneys. 
Furthermore, small roof shapes such as dormers, which are generally unsuitable for installing PV, are partly represented as separate roof polygons in the LOD4 dataset.
For this reason, we convert all surfaces smaller than $8m^2$ are converted to superstructures. 
As the superstructure dataset has been derived from LiDAR data, building-integrated objects such as windows or already existing solar panels are not considered. We thus use only the roof superstructure data as represented in the SITG dataset, which is available for 37.7K roofs in the Canton of Geneva. Roofs without any superstructure information are excluded 

The details of all building datasets used here are summarized in Table~\ref{tab:buildings}.

\begin{table}[tb]
\centering
\footnotesize
\begin{tabular} {lllll} %% {\textwidth}
\hline
\textbf{Data} & \textbf{Coverage} & \textbf{Spatial res.} & \textbf{Creation} & \textbf{Source}\\
\hline 
Roof surfaces  & Switzerland & Rooftop (Vectors) & 2010-2016 & Sonnendach.ch \cite{klauser_solarpotentialanalyse_2016} \\
Register of buildings  & Switzerland & Buildings (Points) & 2015 & SwissStat \cite{federal_statistics_office_federal_2015} \\
Roof superstructures  & Geneva Canton & Rooftop (Vectors) & 2005-2011 & SITG \cite{sitg_superstructures_2019} \\
\hline
\end{tabular}
\caption{Building data used in the study}
\label{tab:buildings}
\end{table}

\textbf{Topographic Landscape Model (TLM).}
The TLM used in this work is obtained from swisstopo \cite{swisstopo_swisstlm3d_2018} and contains a detailed 3D representation of various landscape objects in vector form. 
It has an accuracy of $0.2-1.5m$ for well-defined objects such as roads or buildings, and of $1 - 3m$ for other landscape features such as forests \cite{swisstopo_swisstlm3d_2018}. 
To account for the inaccuracies of the TLM, we apply a tolerance buffer of $1m$ to all objects listed in Section~\ref{GIS}. 
Additionally, roads and railways are buffered by their widths. 
\nomenclature[A]{TLM}{Topographic Landscape Model}


\subsection{Digital Elevation models}
\textbf{Digital Terrain Model.} Two types of surface datasets are used for the projection of PV potential: A Digital Terrain Model (DTM) and a Digital Surface Model (DSM) of Switzerland. The DTM is a high-resolution surface model, without considering vegetation and constructions. It is available at national scale as pixels of $(2\times2)m^2$ resolution. The DTM is derived from Light Detection and Ranging (LiDaR) data that was collected in the period from 2000-2008 and has been updated from 2010-2016.

\textbf{Digital Surface Model.} The DSM is a complete model of the landscape, including all visible landscape elements. As the DTM it is available  in $(2\times2)m^2$ resolution from LiDaR data that was collected in the period from 2000-2008, but contrary to the DSM is has not been updated. More recent DSMs have however been created for individual cantons in Switzerland at a resolution of $(0.5\times0.5)m^2$. 

In this study, we use the higher-resolution and more updated DSM in the canton of Geneva to improve the estimation of shading effects and the skyview factor, as described in Sections \ref{shade} and \ref{svf}. This has two-fold reasons. Firstly, the spatial resolution of the updated DSM is 16 times higher, and secondly, the newer period of construction accounts for some buildings which have been built after the completion of the first DSM. An analysis of the RBD shows that 9.32\% of buildings in Switzerland have been constructed in the period of 2006-2015, corresponding approximately to the time span between the two studies. In Geneva, this fraction is comparable (7.28\%).

\nomenclature[A]{DTM}{Digital Terrain Model}
\nomenclature[A]{DSM}{Digital Surface Model}
\nomenclature[A]{LiDaR}{Light Detection and Ranging}

\begin{table}[b]
\centering
\footnotesize
\begin{tabular} {lllll} %% {\textwidth}
\hline
\textbf{Data} & \textbf{Coverage} & \textbf{Spatial res.} & \textbf{Creation} & \textbf{Source}\\
\hline 
DTM  & Switzerland & $(2\times2) m^2$ & 2010-2016 & SwissTopo \cite{swisstopo_swissalti3d_2017} \\
DSM  & Switzerland & $(2\times2) m^2$ & 2000-2008 & SwissTopo \cite{swisstopo_dsm_2005} \\
DSM  & Geneva Canton & $(0.5\times0.5) m^2$ & 2013 & SITG \cite{sitg_mns_2018} \\
\hline
\end{tabular}
\caption{Digital Elevation Models used in the study}
\label{tab:surface}
\end{table}

\section{Meteorological data}

\subsection{Solar radiation}
% SOURCE: PV paper (original draft)
In the context of this study, satellite data for global horizontal radiation and direct beam radiation provided by the Swiss Federal Office of Meteorology and Climatology (MeteoSwiss) is used \cite{stockli_heliomont_2017}. The radiation describes the total solar power at the earth's surface, given in $W/m^2$. The solar energy is referred to as irradiation, given in $Wh/m^2$. The data are available as hourly values for the period from 2004-2015 on a longitude-latitude grid of 1.25 degree minutes, equivalent to approximately $(1.6 \times 2.3)km^2$. Satellite data is preferred over data from measurement stations as it provides a better spatial coverage with an increased spatial resolution and has a very low missing data ratio ($<1\%$).

The datasets have been derived from Meteosat Second Generation (MSG) satellite observations using the Heliomont algorithm~\cite{stockli_heliomont_2017}. It was developed to improve the quality of the results, particularly in Switzerland's Alpine territories. 
\citet{ineichen_long_2014} performed a comprehensive validation of various satellite-based products against measurement data, and found a negligible bias of the hourly satellite data across 18 measurement stations. 
The standard deviation for hourly global and direct radiation is 19\% and 39\%, respectively.

We average the 12 years of satellite data in order to obtain an average year in hourly resolution, i.e. $12 \times 365$ time steps for $11,243$ satellite pixels. This reduces the variability of the radiation data, and furthermore allows the estimation of long-term PV potential without bias due to extreme meteorological events of a specific year.

\subsection{Temperature}

\begin{table}[tb]
\centering
\footnotesize
\begin{tabular}{lllll} % {X[-1,l] X[-1,l] X[-1,l] X[-1,l] X[-1,l]} %% {\textwidth}
\hline
\textbf{Data}               & \textbf{Spatial res.}        & \textbf{Time} & \textbf{Range} & \textbf{Source} \\
\hline 
Global horizontal radiation & $1.25$ deg. min.\footnotemark & hourly              & 2004-2015          &  MeteoSwiss \cite{stockli_daily_2013} \\
Direct horizontal radiation & $1.25$ deg. min.\footnotemark[\value{footnote}] & hourly              & 2004-2015          &  MeteoSwiss \cite{stockli_daily_2013} \\
Surface albedo              & $1.25$ deg. min.\footnotemark[\value{footnote}] & daily               & 2004-2015          &  MeteoSwiss \cite{stockli_daily_2013} \\
Maximum temperature         & $1 km^2$            & daily               & 2004-2015          &  MeteoSwiss \cite{meteoswiss_daily_2017} \\              
\hline
\end{tabular}
\caption{Meteorological data used in the study.}
\label{tab:meteo}
\end{table}

\footnotetext{Deg. min. denotes degree minutes on a longitude-latitude grid. 1.25 deg. min. corresponds to approximately $(1.6 \times 2.3)km^2$.}

\section{Subsurface data}
\label{data_geo}

The geothermal cadastres, provided by ASIT-VD \cite{asit_vd_cadastre_2019-1} and SITG \cite{sitg_cadastre_2019}, contain information on (i) restriction zones for geothermal installations within the case study area, (ii) thermal ground properties, and (iii) over 4,400 existing installations with $\sim 10,800$ BHEs.

\subsection{Geological characteristics}
\textbf{Swiss Geological Atlas (GK500).}

\textbf{Depth of unconsolidated deposits.}

\subsection{Thermal ground properties}

\textbf{Thermal conductivity.}

\textbf{Thermal diffusivity.}
The ground data provided in the cadastres include the thermal conductivity ($\lambda$) and the heat capacity ($\rho C$, Geneva only), from which the diffusivity is obtained as $\alpha = \lambda / \rho C$ \cite{pahud_geothermal_2002}.
The $\lambda$ and $\rho C$ are computed as the average of the ground properties of each rock layer weighted by % weighted
its thickness, which is given by 3D models of the subsurface \cite{groupe_de_travail_pgg_evaluation_2011-1}.
The ground data exists for depths in the range of $50-300m$ as pixels of $50m$ spatial resolution in all 3 dimensions. 
Constrained by this spatial resolution and $H_{max}$, we obtain four scenarios for the borehole depth $H$ (see Table ~\ref{tab:data}).
To obtain $\lambda$ and $\rho C$ for all scenarios of $H$ in the entire study region, we compute any missing values as weighted averages using
tabulated data \cite{groupe_de_travail_pgg_evaluation_2011-1, sia_sondes_2010}. 
The resulting maps for $\lambda$ and $\alpha$ for a borehole depth of $H = 100m$ are shown in Fig. \ref{fig:lambda} and \ref{fig:alpha}.

\textbf{Ground temperature.}
The data for $T_0$ has been provided by \citet{assouline_machine_2019} for pixels of $200m$ resolution.
The data is estimated from ground measurements using Machine Learning \cite{assouline_machine_2019}.
To minimise the impact of the built environment on these measurements, we use the annual average ground temperature at a depth of $1m$.
Its interpolated values at the resolution of $\lambda$ and $\alpha$ is shown in Fig.~\ref{fig:T0}.
The temperature gradient in the Swiss plateau is approximately constant at $0.03K/m$ \cite{sia_sondes_2010}.

\subsection{Restrictions for geothermal installations}

The restriction zones are divided into three categories, shown in Fig.~\ref{fig:lims}, which indicate whether the installation of BHEs is permitted, limited, or prohibited.
Prohibited zones are excluded from the study. 
The permitted and limited zones are overlaid with information from existing installations to derive $H_{max}$, as no data on the allowed drilling depth is available.
Based on these considerations,
we choose $H_{max}$ as the $75^{th}$ percentile of the depth of the existing BHEs, yielding values of $200m$ and $150m$ for the "permitted" and "limited" zones, respectively. 
