\chapter{Large-scale datasets for Switzerland}
\label{data}

% SOURCE: PV paper (original draft)
Using big data mining techniques for the estimation of large-scale RPV potential requires the availability of accurate and high-resolution environmental and building datasets. Spatial and temporal resolutions of these data are rapidly increasing, but the highest-quality information is frequently not available for a large spatial coverage. Switzerland has been selected as case study area for this approach, as large amounts of data at different resolutions and spatial coverage exist for solar radiation, the building stock and digital surface models.

\section{Building and landscape data}

\subsubsection{Building footprints}
% SOURCE: PV paper (original draft)
We compute the RPV potential per roof surface, based on a national-scale dataset of building rooftops.
The dataset contains 9.6M vector polygons which are derived from a national 3D building model (LOD~2) by the Swiss Federal Office of Topography (SwissTopo) \cite{klauser_solarpotentialanalyse_2016}.
The roof polygons represent the roofs of 3.7M buildings in Switzerland and contain information  on the tilt, aspect and tilted area of each roof.
We combine this information with Switzerland's register of buildings and dwellings (RBD), which contains information on the building's floor area, construction period, number of floors and building type (e.g. residential/industrial). 

In the Canton of Geneva, where detailed city GML data (LOD~4) exists, an additional dataset of roof superstructures is available through Geneva's geographic information system (SITG)~\cite{sitg_superstructures_2019}. %NEED TO ADD SOURCE
These are vector polygons which represent objects on rooftops such as dormers and chimneys. 
Furthermore, small roof shapes such as dormers, which are generally unsuitable for installing PV, are partly represented as separate roof polygons in the LOD4 sataset.
For this reason, all surfaces smaller than $8m^2$ are converted to superstructures in this dataset. 

As the superstructure dataset has been derived from LiDAR data, building-integrated objects such as windows or already existing solar panels are not considered. They can only be detected by performing edge detection or image recognition, as performed by \citet{mainzer_assessment_2017}. This assessment is beyond the scope of the current work. Instead, we use only the roof superstructure data as represented in the SITG dataset, which is available for 37.7K roofs in the Canton of Geneva.

The details of all building datasets used here are summarized in Table~\ref{tab:buildings}.

\begin{table}[tb]
\centering
\footnotesize
\begin{tabular} {lllll} %% {\textwidth}
\hline
\textbf{Data} & \textbf{Coverage} & \textbf{Spatial res.} & \textbf{Creation} & \textbf{Source}\\
\hline 
Roof surfaces  & Switzerland & Rooftop (Vectors) & 2010-2016 & Sonnendach.ch \cite{klauser_solarpotentialanalyse_2016} \\
Register of buildings  & Switzerland & Buildings (Points) & 2015 & SwissStat \cite{federal_statistics_office_federal_2015} \\
Roof superstructures  & Geneva Canton & Rooftop (Vectors) & 2005-2011 & SITG \cite{sitg_superstructures_2019} \\
\hline
\end{tabular}
\caption{Building data used in the study}
\label{tab:buildings}
\end{table}

\subsubsection{Digital Elevation models}

Two types of surface datasets are used for the projection of PV potential: A Digital Terrain Model (DTM) and a Digital Surface Model (DSM) of Switzerland. The DTM is a high-resolution surface model, without considering vegetation and constructions, while the DSM is a complete model of the landscape, including all visible landscape elements.
At national scale, both datasets are available in $(2\times2)m^2$ resolution and are derived from Light Detection and Ranging (LiDAR) data that was collected in the period from 2000-2008. The DTM has been updated from 2010-2016, while the DSM remained unchanged. More recent DSMs have however been created for individual cantons in Switzerland at a resolution of $(0.5\times0.5)m^2$, as summarized in Table~\ref{tab:surface}.

\begin{table}[b]
\centering
\footnotesize
\begin{tabular} {lllll} %% {\textwidth}
\hline
\textbf{Data} & \textbf{Coverage} & \textbf{Spatial res.} & \textbf{Creation} & \textbf{Source}\\
\hline 
DTM  & Switzerland & $(2\times2) m^2$ & 2010-2016 & SwissTopo \cite{swisstopo_swissalti3d_2017} \\
DSM  & Switzerland & $(2\times2) m^2$ & 2000-2008 & SwissTopo \cite{swisstopo_dsm_2005} \\
DSM  & Geneva Canton & $(0.5\times0.5) m^2$ & 2013 & SITG \cite{sitg_mns_2018} \\
\hline
\end{tabular}
\caption{Digital Elevation Models used in the study}
\label{tab:surface}
\end{table}

In this study, we use the higher-resolution and more updated DSM in the canton of Geneva to improve the estimation of shading effects and the skyview factor, as described in Sections \ref{shade} and \ref{svf}. This has two-fold reasons. Firstly, the spatial resolution of the updated DSM is 16 times higher, and secondly, the newer period of construction accounts for some buildings which have been built after the completion of the first DSM. An analysis of the RBD shows that 9.32\% of buildings in Switzerland have been constructed in the period of 2006-2015, corresponding approximately to the time span between the two studies. In Geneva, this fraction is comparable (7.28\%).


\section{Meteorological data}

% SOURCE: PV paper (original draft)
In the context of this study, satellite data for global horizontal radiation and direct beam radiation provided by the Swiss Federal Office of Meteorology and Climatology (MeteoSwiss) is used \cite{stockli_heliomont_2017}. The radiation describes the total solar power at the earth's surface, given in $W/m^2$. The solar energy is referred to as irradiation, given in $Wh/m^2$. The data are available as hourly values for the period from 2004-2015 on a longitude-latitude grid of 1.25 degree minutes, equivalent to approximately $(1.6 \times 2.3)km^2$. Satellite data is preferred over data from measurement stations as it provides a better spatial coverage with an increased spatial resolution and has a very low missing data ratio ($<1\%$).

The datasets have been derived from Meteosat Second Generation (MSG) satellite observations using the Heliomont algorithm~\cite{stockli_heliomont_2017}. It was developed to improve the quality of the results, particularly in Switzerland's Alpine territories. 
\citet{ineichen_long_2014} performed a comprehensive validation of various satellite-based products against measurement data, and found a negligible bias of the hourly satellite data across 18 measurement stations. 
The standard deviation for hourly global and direct radiation is 19\% and 39\%, respectively.

We average the 12 years of satellite data in order to obtain an average year in hourly resolution, i.e. $12 \times 365$ time steps for $11,243$ satellite pixels. This reduces the variability of the radiation data, and furthermore allows the estimation of long-term PV potential without bias due to extreme meteorological events of a specific year.

\begin{table}[tb]
\centering
\footnotesize
\begin{tabular}{lllll} % {X[-1,l] X[-1,l] X[-1,l] X[-1,l] X[-1,l]} %% {\textwidth}
\hline
\textbf{Data}               & \textbf{Spatial res.}        & \textbf{Time} & \textbf{Range} & \textbf{Source} \\
\hline 
Global horizontal radiation & $1.25$ deg. min.\footnotemark & hourly              & 2004-2015          &  MeteoSwiss \cite{stockli_daily_2013} \\
Direct horizontal radiation & $1.25$ deg. min.\footnotemark[\value{footnote}] & hourly              & 2004-2015          &  MeteoSwiss \cite{stockli_daily_2013} \\
Surface albedo              & $1.25$ deg. min.\footnotemark[\value{footnote}] & daily               & 2004-2015          &  MeteoSwiss \cite{stockli_daily_2013} \\
Maximum temperature         & $1 km^2$            & daily               & 2004-2015          &  MeteoSwiss \cite{meteoswiss_daily_2017} \\              
\hline
\end{tabular}
\caption{Meteorological data used in the study.}
\label{tab:meteo}
\end{table}

\footnotetext{Deg. min. denotes degree minutes on a longitude-latitude grid. 1.25 deg. min. corresponds to approximately $(1.6 \times 2.3)km^2$.}

\section{Subsurface data}

\section{Energy demand data}
